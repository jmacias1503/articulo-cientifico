La realidad virtual no es una tecnología novedosa actualmente. El primer casco de realidad virtual fue desarrollado en la década de 1970 en la Universidad de Utah por Daniel Vickers. Este casco, que constaba de dos pantallas, permitía a los usuarios el poder explorar entornos virtuales al momento de girar la cabeza. Poco después, se introdujo una interfaz llamada "DataGlove" en 1982, que lograba seguir el movimiento de la mano y los dedos y lo transmitía a una computadora. El término "Realidad Virtual" fue propuesto en los Estados Unidos en la década de 1980 por Jaron Lanier. A pesar de la fiebre tecnológica de la realidad virtual en la década de 1990, en ese momento, los fabricantes se enfrentaron a numerosas limitaciones que impidieron la aceptación generalizada por parte del público. Cuando realmente empezó a evolucionar más esta tecnología es a partir de 2012. Se introdujeron nuevas mejoras tanto en términos tecnológicos como comerciales, que han hecho que la experiencia de utilizar la realidad virtual sea lo suficientemente cómoda como para atraer la atención de un público más amplio. Los principales avances son la accesibilidad y el rendimiento de las computadoras y los teléfonos inteligentes; esto junto con el apoyo del acceso a Internet y el aumento de la velocidad de las conexiones, ha generado una transformación. Según el banco de inversión Citi, se esperaba que el valor del mercado de la realidad virtual, alcanzará los 200 billones de dólares en 2020. El uso de esta tecnología se refleja en las múltiples opciones de usos, las cuales pueden incluir la simulación de procedimientos médicos, el diseño arquitectónico, las visitas a museos virtuales, métodos de terapia y otras formas de aprendizaje. Algunas ya consolidadas, como es el uso en simuladores de vuelo. Michael Bodekaer quiere progresar esta tecnología con un laboratorio virtual, el cual brinda a científicos, la oportunidad de llevar a cabo pruebas y experimentos sin riesgos físicos, con menores costos y con mejores resultados. \cite{elmqaddem2019augmented}.
