La Realidad Virtual (RV) está en un punto en el cual puede llegar a revolucionar la educación, ofreciendo a los estudiantes una experiencia de aprendizaje inmersiva que trasciende los enfoques convencionales. Este artículo explorará en profundidad cómo la RV está transformando las aulas y cómo podría cambiar el futuro de la educación. Mientras la RV permite a los estudiantes adentrarse en entornos tridimensionales que simulan situaciones del mundo real o conceptos abstractos, esto mismo ofrece ventajas notables para mejorar la retención y comprensión del conocimiento adquirido. Además, la colaboración y el aprendizaje se fomentan a través de proyectos virtuales. Estos desafíos se analizan mediante el apoyo de recolección de información de otras fuentes bibliográficas adquiridas en bases de datos como Google Académico (258,000 articulos encontrados, 7 utilizados) y Elsevier (1858 articulos encontrados, 13 utilizados), considerando la calidad de los mismos. Una de las fuentes principales de información ha sido la revista \textit{Computers \& Education: X Reality}, donde se habla a fondo acerca de las tecnologías que pueden ser implementadas y ayudar en la educación, utilizando las palabras clave \textit{virtual reality, education} (también traducida respectivamente al español). Con base en el análisis hecho de las fuentes de información, la RV puede llegar a ser una gran herramienta en la educación, mas no puede reemplazar partes importantes del sistema educativo tradicional, ya que se puede llegar a adicciones o sobre estimulación donde la experiencia de aprendizaje puede llegar a ser deslindada.
