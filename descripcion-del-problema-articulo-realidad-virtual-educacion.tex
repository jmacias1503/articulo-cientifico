Una de las vertientes de los motivos de investigación del tema ha sido la repetitiva estructura del sistema educativo, en la cual, se llega a caer en muchas estructuras que son repetitivas, y, que si se exploraran de distintas maneras, se podría llegar a mejores resultados
En algunas partes del mundo, todavía se cree que que los aparatos de realidad virtual y aumentada solo puede llegar al ámbito de los videojuegos, cuando todavía hay un enorme horizonte que se puede explorar en todos los ámbitos posibles, en esta ocasión, la educación desde todos los niveles, desde la básica hasta superior.

El acercamiento a la realidad virtual ha sido muy lento, debido a supuestas complicaciones de salud que puede generar, al igual que los tabús que se tienen de esta. Ademas, en distintas profesiones en las que gran parte del aprendizaje que es empírico, en ciertos casos se aprende en situaciones de riesgo, un acercamiento desde la realidad virtual podría mantener es mismo aprendizaje, sin que hubiera peligro alguno.
