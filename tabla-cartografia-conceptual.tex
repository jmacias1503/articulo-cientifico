\begin{table}[!h]
   \caption{Cartografía conceptual}
   \begin{tabular}{p{3.5cm}|p{4.5cm}|p{4.2cm}}
      Eje de análisis & Pregunta Central & Componentes\\
      \hline
      Noción & ¿Cuál es la definición de realidad virtual en el ámbito de la educación, su popularización y en que se destaca? & Definición de términos. Historia. Importancia.\\
      Categorización & ¿A qué clase mayor pertenece el concepto de realidad virtual? & Clase inmediata. Clase que sigue.\\
      Caracterización & ¿Cuáles son las características centrales de la realidad virtual? & Características en base a la noción y la categorización. Explicación de características.\\
      Diferenciación & ¿Cómo se diferencia la realidad virtual de otros conceptos similares? & Definición de los conceptos. Diferencias entre conceptos.\\
      Clasificación & En que rubros se puede clasificar las distintas aplicaciones de la RV & Aplicaciones de la RV en distintas áreas de estudio\\
      Vinculación & Como podemos ligar los beneficios propuestos a dinámicas y cuantificaciones& Muestra de resultados. Resultados a diferencia de otros métodos\\
      Metodología & Que métodos se pueden emplear para probar el funcionamiento de la RV?& Muestra de resultados. Emplearon de la RV a la educación.\\
      Ejemplificación & Como se puede aplicar la RV en el ámbito de la educación? & Casos de aplicación de la RV en una muestra real\\
   \end{tabular}
\end{table}
