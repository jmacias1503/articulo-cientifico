\section{Resultados}

Después del análisis de las fuentes de información, la RV es una tecnología que podría ayudar a mejorar el sistema de educación actual, ya que en este hay numerosas ventajas y desventajas al implementarlo en un modelo educativo.
%\begin{table}[H]
%   \caption{Uso de RV respecto a cada fuente de información}
%   \label{tab:validacionrv}
%   \begin{center}
%      \begin{tabular}{p{3cm}|p{10cm}}
%         \hline
%         Fuente & Conclusión\\
%         \hline
%         \citetitle{ZAMMIT2023100035} & Utilizar la RV como una herramienta, mas no utilizarla totalmente\\
%         \citetitle{GUERRERO2022100002} & Implementar la RV como un nuevo sistema\\
%         \citetitle{SHIM2023100010} & La RV puede funcionar para conocimientos y habilidades especificas, pero no todas\\
%         \hline
%      \end{tabular}
%   \end{center}
%\end{table}
En base a los resultados, es clara la visión en que la RV podría llegar a mejorar la educación, sin embargo, esta no es viable reemplazarla totalmente, ya que puede haber casos donde la sobre estimulación puede llegar a deslindar desde la ruta hecha para el aprendizaje. Para esta hay que mantener una moderacion, ya que puede llegar a ser perjudicial el solo trabajar con RV en distintas areas. Asi como es beneficioso, tiene desventajas que pueden llegar a ser decisivas.
