Los estudios vistos en \cite{SHIM2023100010}, han mostrado que el modelo de educación en realidad virtual ha mejorado sensibilidad moral durante este programa, mientras que el juicio moral se ha mantenido igual. El modelo se basaba en actividad en RV por 10 minutos por estudiante, en un total de 60 minutos, donde se hacían actividades de inmersión, interacción, imaginación y aprendizaje colaborativo para después una discusión de clase de 40 minutos en la cual se hacían reflexiones, se empleaba el pensamiento lógico, y también la empatia. La actividad en RV consistía en que el niño que jugaba era el personaje principal, y los otros 5 estudiantes discutían. El contenido del recorrido consistía de 10 etapas, donde el equipo de investigación colaboro con \textit{Studio Coin}, una compañía creadora de contenido educacional utilizando RV y realidad aumentada. El utilizar esto para poder trabajar la moral no es efectivo, ya que solo se trabaja la sensibilidad moral, y no el juicio moral.
\begin{table}[H]
   \caption{Estadisticas descriptivas de los cuatro componentes de realidad virtual (n = 162)}
   \label{tab:statsmoralvr}
   \begin{center}
      \begin{tabular}{p{2cm} p{2cm} p{2cm} p{2cm} p{2cm}}
         \hline
         Variable & Promedio & Desviación Estándar & Valor Mínimo & Valor Máximo\\
         \hline
         IA & 4.605 & 0.517 & 3 & 5\\
         IG & 4.282 & 0.654 & 2 & 5\\
         IM & 4.304 & 0.546 & 2.2 & 5\\
         CL & 4.280 & 0.414 & 1.8 & 5\\
         \hline
      \end{tabular}
   \end{center}
      \textit{Nota. }IA: interacción; IG: imaginación; IM: inmersión; CL: Aprendizaje Colaborativo. Tabla adaptada y traducida de \cite{SHIM2023100010}
\end{table}
