En tiempos actuales, se ha notado como la realidad virtual ha empezado a formar parte de muchas áreas del trabajo y de la vida, como lo pueden ser la medicina, entretenimiento, educación, etc. Numerosos estudios han utilizado la realidad virtual en la educación, y han desarrollado aplicaciones, por ejemplo, donde se apoya a los estudiantes a entender conceptos de la computación como lo puede ser un \textit{bubble sort}. Al igual que este ejemplo, hay otros numerosos donde se prueba el poder de la realidad virtual en la educación, como lo puede ser también en el videojuego \textit{VR-OCKS}, que esta apuntado a ensenar los principios básicos de la programación. En base a los diversos estudios hechos, se llego con una ecuación para determinar la distancia objetiva para el aprendizaje de un concepto.
\begin{equation}
    OD_i = \left(\frac{S_i - C_i}{T_i}\right) \left(\sqrt{(C_i - S_i)^2}\right) 
   \label{eq:distancia-objetiva}
\end{equation}
donde:
\begin{itemize}
   \item $ OD_i $ = Distancia objetiva para una muestra $i$ de un \textit{learning object}
   \item $ S_i $ = Puntaje satisfactorio para una muestra $i$ de un \textit{learning object}
   \item $ C_i $ = Puntaje actual para una muestra $i$ de un \textit{learning object}
   \item $ T_i $ = Puntaje total para una muestra $i$ de un \textit{learning object}
   \item $ i $ = 1,2,3...
\end{itemize}
Los descubrimientos empíricos encontrados han contribuido un conocimiento significante en como una aplicación basada en realidad virtual puede facilitar el proceso de aprendizaje del estudiantado. \cite{OYELERE2023100016}
