Aunque el uso de la realidad virtual ha tenido muchas ventajas, algunos investigadores han reportado que este no es mas efectivo que cualquier otro método tradicional en algunos beneficios, como lo puede ser el conocimiento y el rendimiento de puntaje. De un total de 7 estudios reportados, han dado que el rendimiento de puntaje que la educación implementada con realidad virtual mejoró el conocimiento de los participantes de manera más efectiva. Respecto a la satisfacción de aprendizaje, no hubo diferencia entre los participantes utilizando realidad virtual y los participantes utilizando métodos tradicionales. En términos de confianza, no se halló alguna diferencia entre las condiciones de control y experimentales. La RV no pudo mejorar la confianza de los participantes de manera más efectiva que en las condiciones de control. Existe el riesgo de que exista el sesgo se presenta en la situación. En un meta análisis, los resultados sugieren que la RV no fue mejor que los otros métodos de educación a la hora del tiempo de rendimiento. En los 12 estudios evaluados, tienen diferentes intervenciones en los grupos de control, lo cual puede causar una gran disfunción en estos. \cite{chen2020effectiveness}.
