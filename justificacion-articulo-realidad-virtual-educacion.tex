\subsection{Justificación}
En algunas partes del mundo, todavía se cree que que los aparatos de realidad virtual y aumentada solo puede llegar al ámbito de los videojuegos, cuando todavía hay un enorme horizonte que se puede explorar en todos los ámbitos posibles, en esta ocasión, la educación desde todos los niveles, desde la básica hasta superior.

La investigación sobre este tema es debido a la tecnología que esta aumentando con creces, aparte que la realidad virtual es  una idea que se materializo apenas y es muy joven, e incluso con esto, ya se han mostrado beneficios utilizando en el ámbito de la educación, saliendo de los sistemas tradicionales de este.

Utilizar la realidad virtual como una herramienta de educación puede llegar a ser muy beneficioso, debido a las posibilidades que se puede llegar con esta, desde modelados 3D que se pueden visualizar con las perspectiva de quien la vea, hasta simulaciones con variables realistas en las cuales se pueden simular casos que llegan a suceder en la vida real, sin tener los riesgos de estar en una.
