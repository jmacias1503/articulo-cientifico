\subsection{Cartografía Conceptual}

\subsubsection{Noción}

“Realidad virtual” proviene de “realidad” (lo real) y “virtual” (lo simulado). Sugiere una representación artificial con potencial realismo. El término se popularizó en la década de 1980, usado inicialmente en informática.La realidad virtual en educación crea entornos de aprendizaje inmersivos que imitan experiencias del mundo real. Destaca la inmersión, interactividad y adaptabilidad a las necesidades del estudiante \parencite{zheng1998virtual}.

\subsubsection{Categorización}

La realidad virtual en la educación se encuentra dentro de la categoría de “tecnología educativa” o “aprendizaje digital”. Las categorías cercanas incluyen “aprendizaje inmersiva” y “simulación educativa”. Su categorías posibles podrían ser “entornos de laboratorio virtual” o “aulas virtuales”. \parencite{barahona2019cospaces, marin2022realidad}

\subsubsection{Caracterización}

La realidad virtual en educación se caracteriza por la inmersión total del estudiante en un entorno virtual interactivo. Esta inmersión implica que los usuarios se sienten completamente inmersos en el entorno virtual, lo que les permite interactuar y aprender de manera efectiva en un contexto simulado. \parencite{zamudio2021realidad}

\subsubsection{Diferenciación}

Conceptos similares a la realidad virtual en la educación incluyen la “realidad aumentada” y el “aprendizaje en línea”. La “realidad aumentada” superpone elementos virtuales en el mundo real, mientras que el “aprendizaje en línea” implica la enseñanza y el aprendizaje a través de Internet. La realidad virtual sumerge a los usuarios en entornos virtuales completamente simulados, mientras que la realidad aumentada añade elementos virtuales al mundo real. El aprendizaje en línea se refiere a la educación basada en la web sin necesidad de inmersión en entornos virtuales \parencite{garcia2020}

\subsubsection{Clasificación}

La RV ha sido un medio en el cual múltiples áreas del estudio han sido adaptadas por esta, como lo puede ser en la medicina, donde ha habido casos de entrenamiento para enfermeros/as para determinar pacientes en potencia de tener COVID-19 \parencite{GUERRERO2022100002}

\subsubsection{Vinculación}

Se notaron en los resultados de \textcite{SHIM2023100010} que la RV si puede crear una diferencia a la hora del aprendizaje, mas no es abismal, y no trata todos los aspectos de la educación moral. Al igual, en estudios hacia enfermeras se notaron mucho mas competentes aquellas con simulaciones, que, al ser evaluadas antes y después de tomar una simulación de alta fidelidad, hubo un gran incremento en los promedios de al menos 20.67 puntos sobre 100. \parencite{GUERRERO2022100002}

\subsubsection{Metodología}
Utilizando métodos introductorios a la realidad virtual, como lo puede ser la realidad aumentada, se pueden tomar pruebas que incluyan rubros como la física, y hacer sesiones de tutoría remotas. Con ello, se pueden visualizar conceptos digitales a nuestra realidad con los que la experiencia de aprendizaje es mas agradable, y múltiples conceptos se pueden materializar en imágenes a tiempo real \parencite{RADU2023100011}



\subsubsection{Ejemplificación}

Un ejemplo aplicado de la RV en la educación son los estudios y experimentos hechos por \textcite{SHIM2023100010}, donde se aplica hacia un grupo de niños donde el rumbo de la investigación es hacia la educación moral, componente importante para el desarrollo del ser humano, y que, se demostró que la sensibilidad moral mostró un incremento significativo, mas no el juicio moral.
