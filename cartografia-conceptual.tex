\subsection{Cartografía Conceptual}

\subsubsection{Noción}

“Realidad virtual” proviene de “realidad” (lo real) y “virtual” (lo simulado). Sugiere una representación artificial con potencial realismo. El término se popularizó en la década de 1980, usado inicialmente en informática. La realidad virtual en educación crea entornos de aprendizaje inmersivos que imitan experiencias del mundo real. Destaca la inmersión, interactividad y adaptabilidad a las necesidades del estudiante \parencite{zheng1998virtual}.

La realidad virtual se puede definir como una inmersi{\'o}n humana a un mundo sint{\'e}tico. Esta tecnolog{\'i}a permite a cada usuario mantenerse en un mundo nuevo, en el cual, hay oportunidades inmensas para tanto el aprendizaje, como el entretenimiento \parencite{elmqaddem2019augmented}.
\subsubsection{Categorización}

La realidad virtual en la educación se encuentra dentro de la categoría de “tecnología educativa” o “aprendizaje digital”. Las categorías cercanas incluyen “aprendizaje inmersiva” y “simulación educativa”. Su categorías posibles podrían ser “entornos de laboratorio virtual” o “aulas virtuales”. \parencite{barahona2019cospaces, marin2022realidad}

\subsubsection{Caracterización}

La realidad virtual en educación se caracteriza por la inmersión total del estudiante en un entorno virtual interactivo. Esta inmersión implica que los usuarios se sienten completamente inmersos en el entorno virtual, lo que les permite interactuar y aprender de manera efectiva en un contexto simulado. \parencite{zamudio2021realidad}

Las ventajas de la realidad virtual para la comunicación en ingles como segundo lenguaje ha probado que su inmersión, interacción, retroalimentación y creación han sede percibidas positivamente y son efectivas para lo que se afronta, la ansiedad, la motivación, confianza en si mismo, la conciencia cultural, la creatividad, y la voluntad de comunicarse. Aunque, los resultados en lo efectivo que puede ser el aprendizaje son aun inconclusos, aunque esto es probablemente causado por la gran carga cognitiva, problemas de equidad, experiencias no gratas, retos tecnológicos, y falta de actividades instrucciones en el ambiente de realidad virtual. \parencite{YUDINTSEVA2023100018}


\subsubsection{Diferenciación}

Conceptos similares a la realidad virtual en la educación incluyen la “realidad aumentada” y el “aprendizaje en línea”. La “realidad aumentada” superpone elementos virtuales en el mundo real, mientras que el “aprendizaje en línea” implica la enseñanza y el aprendizaje a través de Internet. La realidad virtual sumerge a los usuarios en entornos virtuales completamente simulados, mientras que la realidad aumentada añade elementos virtuales al mundo real. El aprendizaje en línea se refiere a la educación basada en la web sin necesidad de inmersión en entornos virtuales \parencite{garcia2020, LOWELL2023100017}

Los resultados entre una educaci\'on hecha por un sistema tradicional a comparaci{\'o}n de uno utilizando la realidad virtual llegan a mostrar distintos resultados,en unos, se denota una mejo experiencia de aprendizaje, mientras que en otros, la mejora no es significativa. \parencite{palma2020realidad, SHIM2023100010, GUERRERO2022100002}

\subsubsection{Clasificación}

La RV ha sido un medio en el cual múltiples áreas del estudio han sido adaptadas por \'esta, como lo puede ser en la medicina, donde ha habido casos de entrenamiento para enfermeros/as para determinar pacientes en potencia de tener COVID-19 \parencite{GUERRERO2022100002}. No solo se excluye a este tipo de temas, sino que se puede aplicar tambi{\'e}n a la educaci{\'o}n b{\'a}sica \parencite{marin2022realidad}.

\subsubsection{Vinculación}

Se notaron en los resultados de \textcite{SHIM2023100010} que la RV s\'i puede crear una diferencia a la hora del aprendizaje, mas no es abismal, y no trata todos los aspectos de la educación moral. Al igual, en estudios hacia enfermeras se notaron mucho m\'as competentes aquellas con simulaciones al ser evaluadas antes y después de tomar una simulación de alta fidelidad. Hubo un gran incremento en los promedios de al menos 20.67 puntos sobre 100. \parencite{GUERRERO2022100002}

\input{tabla-guerrero2022-resultados-simulacion.tex}

\begin{table}[H]
   \caption{Estad\'isticas descriptivas de los cuatro componentes de realidad virtual (n = 162)}
   \label{tab:statsmoralvr}
   \begin{center}
      \begin{tabular}{p{2cm} p{2cm} p{2cm} p{2cm} p{2cm}}
         \hline
         Variable & Promedio & Desviación Estándar & Valor Mínimo & Valor Máximo\\
         \hline
         IA & 4.605 & 0.517 & 3 & 5\\
         IG & 4.282 & 0.654 & 2 & 5\\
         IM & 4.304 & 0.546 & 2.2 & 5\\
         CL & 4.280 & 0.414 & 1.8 & 5\\
         \hline
      \end{tabular}
   \end{center}
      \textit{Nota. }IA: interacción; IG: imaginación; IM: inmersión; CL: Aprendizaje Colaborativo. Tabla adaptada y traducida de \textcite{SHIM2023100010}
\end{table}


\subsubsection{Metodología}
Utilizando métodos introductorios a la realidad virtual, como lo puede ser la realidad aumentada, se pueden tomar pruebas que incluyan rubros como la física, y hacer sesiones de tutoría remotas. Con ello, se pueden visualizar conceptos digitales a nuestra realidad con los que la experiencia de aprendizaje es m\'as agradable, y múltiples conceptos se pueden materializar a imágenes en tiempo real \parencite{RADU2023100011}

\subsubsection{Ejemplificación}

Un ejemplo aplicado de la RV en la educación son los estudios y experimentos hechos por \textcite{SHIM2023100010}, donde se aplica hacia un grupo de niños donde el rumbo de la investigación es hacia la educación moral, componente importante para el desarrollo del ser humano, y que, demostró que la sensibilidad moral mostró un incremento significativo, mas no el juicio moral. Y en otros estudios, \textcite{OJE2023100033} muestra la educaci{\'o}n asistida con RV, y con ello, que en un futuro se puedan desarrollar m{\'a}s contenidos hacia la educaci{\'o}n en ingenier{\'i}a en realidad virtual.

En tiempos actuales, se ha notado como la realidad virtual ha empezado a formar parte de muchas áreas del trabajo y de la vida, como lo pueden ser la medicina, entretenimiento, educación, etc. Numerosos estudios han utilizado la realidad virtual en la educación, y han desarrollado aplicaciones, por ejemplo, donde se apoya a los estudiantes a entender conceptos de la computación como lo puede ser un \textit{bubble sort}. \parencite{OYELERE2023100016} 
