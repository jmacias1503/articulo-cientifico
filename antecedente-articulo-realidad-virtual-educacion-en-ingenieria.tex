Muchos estudiantes de ingeniería encuentran que los conceptos mas abstractos que se ven en esta rama llegan a ser muy intangibles para poder asimilarlos de manera correcta. Los ambientes educacionales de la Realidad Virtual pueden ser beneficiosos para la asimilación de estos conceptos. Muchos estudios han propiciado el uso de esta para ayudar al entendimiento conceptual, resolución de problemas y el entrenamiento de habilidades. Es altamente probable que mas educadores de la ingeniería incrementalmente vayan buscando la ayuda de la Realidad Virtual para propiciar su uso en los campos de la ingeniería. El principio de coherencia propone que el aprendizaje mejora cuando es interesante, pero, cuando hay recursos irrelevantes (detalles seductivos) aumentan la carga cognitiva, y por tanto, impiden un buen aprendizaje. Cuando los estudiantes actúan  en situaciones que les hace usar sus habilidades motrices, activan representaciones mentales que les permiten pensar acerca de conceptos que llegan a ser abstractos.\cite{OJE2023100033}
