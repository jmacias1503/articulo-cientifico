\section{Metodología}
\subsection{Tipo de Estudio}
Para este método se utilizo la cartografía conceptual, método que “sirve para formar y evaluar conceptos esenciales de cada competencia en lo que respecta al saber conocer” \parencite[][p. 16]{tobon2012}. Ademas, se tomo como base la estructura de la metodolgia de \textcite{guzman2020} 
\subsection{Técnica de análisis}
Se utilizaron 8 ejes principales de analisis para clasificar los artículos, denotados anteriormente por \textcite{tobon2012} .
\begin{itemize}
   \item Noción
   \item Categorización
   \item Caracterización
   \item Diferenciación
   \item Clasificación
   \item Vinculación
   \item Metodología
   \item Ejemplificación
\end{itemize}

\begin{table}[!h]
   \caption{Cartografía conceptual}
   \begin{tabular}{p{3.5cm}|p{4.5cm}|p{4.2cm}}
      Eje de análisis & Pregunta Central & Componentes\\
      \hline
      Noción & ¿Cuál es la definición de realidad virtual en el ámbito de la educación, su popularización y en que se destaca? & Definición de términos. Historia. Importancia.\\
      Categorización & ¿A qué clase mayor pertenece el concepto de realidad virtual? & Clase inmediata. Clase que sigue.\\
      Caracterización & ¿Cuáles son las características centrales de la realidad virtual? & Características en base a la noción y la categorización. Explicación de características.\\
      Diferenciación & ¿Cómo se diferencia la realidad virtual de otros conceptos similares? & Definición de los conceptos. Diferencias entre conceptos.\\
      Clasificación & En que rubros se puede clasificar las distintas aplicaciones de la RV & Aplicaciones de la RV en distintas áreas de estudio\\
      Vinculación & Como podemos ligar los beneficios propuestos a dinámicas y cuantificaciones& Muestra de resultados. Resultados a diferencia de otros métodos\\
      Metodología & Que métodos se pueden emplear para probar el funcionamiento de la RV?& Muestra de resultados. Emplearon de la RV a la educación.\\
      Ejemplificación & Como se puede aplicar la RV en el ámbito de la educación? & Casos de aplicación de la RV en una muestra real\\
   \end{tabular}
\end{table}


\subsection{Criterios para la selección de los documentos}

Para esta etapa, se seleccionaron las siguientes partes: palabras clave y bases de datos. Para este trabajo, se eligieron bases de datos que tengan un enfoque científico; en este caso se utilizó Google Académico, Elsevier y SciELO. Los términos de búsqueda implementados fueron “realidad virtual”, “educacion en el aula” y “nuevas tecnologías”.

Cada documento fue seleccionado en base los siguientes criterios:

\begin{enumerate}
   \item Incluir las palabras clave.
   \item Enfocarse en el estudio o análisis del proceso de enseñanza/aprendizaje de la realidad virtual en las aulas.
   \item Tener autor, año y responsable del artículo.
   \item Ser artículos publicados en años recientes.
\end{enumerate}


\subsection{Fases de estudio}

La investigacion se realizo por medio de las siguientes fases:

\begin{itemize}
   \item Fase 1. Se buscaron ls articulos referentes al tema en bases de datos primarias para el tiop de conocimiento, entre ellas Google academico, Elsevier y Scielo.
      \begin{table}[H]
   \caption{Tabla de artículos encontrados, revisados, y utilizados}
   \label{tab:articulos}
      \begin{tabular}{l|l l l}
         \hline
         ~ & Google académico & Elsevier & SciELO\\
         \hline
         Artículos encontrados & 258,000 & 1854 & 30\\
         \hline
         Artículos útiles & 7 & 13 & 2\\
         \hline
      \end{tabular}
\end{table}

   \item Fase 2. Descartar los articulos cuyos criterios no cumplian con los esperados
   \item Fase 3. Realizar la cartografía conceptual
\end{itemize}

%\begin{table}[H]
   \caption{Artículos clasificados en su información}
   \label{tab:articulos-concepto-metod-res-inn}
      \begin{tabular}{|p{3.5cm}|p{3.5cm}|p{3.5cm}|p{3.5cm}|}
         \hline
         Concepto & Metodología & Resultados & innovación\\
         \hline
         \citetitle{rodriguez2021simulacion} & \citetitle{zhan2020augmented} & \citetitle{zamudio2021realidad} & \citetitle{barahona2019cospaces}\\
         \hline
         \citetitle{kaminska2019virtual} & \citetitle{palma2020realidad} & \citetitle{zheng1998virtual} &\\
         \hline
         \citetitle{elmqaddem2019augmented} & \citetitle{jang2021augmented} & \citetitle{chen2020effectiveness} &\\
         \hline
         \citetitle{ZAMMIT2023100035} & \citetitle{marin2022realidad} & \citetitle{OJE2023100033} &\\
         \hline
         \citetitle{YUDINTSEVA2023100018} & \citetitle{GUERRERO2022100002} & \citetitle{LOWELL2023100017} &\\
         \hline
         & \citetitle{RADU2023100011} & &\\
         \hline
         & \citetitle{SHIM2023100010} & &\\
         \hline
         & \citetitle{OYELERE2023100016} & &\\
         \hline
      \end{tabular}
\end{table}

\subsection{Cartografía Conceptual}

\subsubsection{Noción}

“Realidad virtual” proviene de “realidad” (lo real) y “virtual” (lo simulado). Sugiere una representación artificial con potencial realismo. El término se popularizó en la década de 1980, usado inicialmente en informática.La realidad virtual en educación crea entornos de aprendizaje inmersivos que imitan experiencias del mundo real. Destaca la inmersión, interactividad y adaptabilidad a las necesidades del estudiante \parencite{zheng1998virtual}.

La realidad virtual se puede definir como una inmersi{\'o}n humana a un mundo sint{\'e}tico. Esta tecnolog{\'i}a permite a cada usuario mantenerse en un mundo nuevo, en el cual, hay oportunidades inmensas para tanto el aprendizaje, como el entretenimiento\parencite{elmqaddem2019augmented}
\subsubsection{Categorización}

La realidad virtual en la educación se encuentra dentro de la categoría de “tecnología educativa” o “aprendizaje digital”. Las categorías cercanas incluyen “aprendizaje inmersiva” y “simulación educativa”. Su categorías posibles podrían ser “entornos de laboratorio virtual” o “aulas virtuales”. \parencite{barahona2019cospaces, marin2022realidad}

\subsubsection{Caracterización}

La realidad virtual en educación se caracteriza por la inmersión total del estudiante en un entorno virtual interactivo. Esta inmersión implica que los usuarios se sienten completamente inmersos en el entorno virtual, lo que les permite interactuar y aprender de manera efectiva en un contexto simulado. \parencite{zamudio2021realidad}

Las ventajas de la realidad virtual para la comunicación en ingl\'es como segundo lenguaje ha probado que su inmersión, interacción, retroalimentación y creación han sido percibidas positivamente y son efectivas para lo que se afronta, la ansiedad, la motivación, confianza en s\'i mismo, la conciencia cultural, la creatividad, y la voluntad de comunicarse. Aunque, los resultados en lo efectivo que puede ser el aprendizaje son aun inconclusos, aunque esto es probablemente causado por la gran carga cognitiva, problemas de equidad, experiencias no gratas, retos tecnológicos, y falta de actividades e instrucciones en el ambiente de realidad virtual. \parencite{YUDINTSEVA2023100018}


\subsubsection{Diferenciación}

Conceptos similares a la realidad virtual en la educación incluyen la “realidad aumentada” y el “aprendizaje en línea”. La “realidad aumentada” superpone elementos virtuales en el mundo real, mientras que el “aprendizaje en línea” implica la enseñanza y el aprendizaje a través de Internet. La realidad virtual sumerge a los usuarios en entornos virtuales completamente simulados, mientras que la realidad aumentada añade elementos virtuales al mundo real. El aprendizaje en línea se refiere a la educación basada en la web sin necesidad de inmersión en entornos virtuales \parencite{garcia2020, LOWELL2023100017}

Los resultados entre una educacion hecha por un sistema tradicional a comparaci{\'o}n de uno utilizando la realidad virtual llegan a mostrar distintos resultados, los cuales arrojan resultados completamente distintos, en unos, se denota una mejo experiencia de aprendizaje, mientras que en otras, la mejora no es significativa. \parencite{palma2020realidad, SHIM2023100010, GUERRERO2022100002}

\subsubsection{Clasificación}

La RV ha sido un medio en el cual múltiples áreas del estudio han sido adaptadas por esta, como lo puede ser en la medicina, donde ha habido casos de entrenamiento para enfermeros/as para determinar pacientes en potencia de tener COVID-19 \parencite{GUERRERO2022100002}. A esto, no solo se excluye a este tipo de temas, sino que se puede aplicar tambi{\'e}n a la educaci{\'o}n b{\'a}sica \parencite{marin2022realidad}.

\subsubsection{Vinculación}

Se notaron en los resultados de \textcite{SHIM2023100010} que la RV si puede crear una diferencia a la hora del aprendizaje, mas no es abismal, y no trata todos los aspectos de la educación moral. Al igual, en estudios hacia enfermeras se notaron mucho mas competentes aquellas con simulaciones, que, al ser evaluadas antes y después de tomar una simulación de alta fidelidad, hubo un gran incremento en los promedios de al menos 20.67 puntos sobre 100. \parencite{GUERRERO2022100002}

\subsubsection{Metodología}
Utilizando métodos introductorios a la realidad virtual, como lo puede ser la realidad aumentada, se pueden tomar pruebas que incluyan rubros como la física, y hacer sesiones de tutoría remotas. Con ello, se pueden visualizar conceptos digitales a nuestra realidad con los que la experiencia de aprendizaje es mas agradable, y múltiples conceptos se pueden materializar en imágenes a tiempo real \parencite{RADU2023100011}

\subsubsection{Ejemplificación}

Un ejemplo aplicado de la RV en la educación son los estudios y experimentos hechos por \textcite{SHIM2023100010}, donde se aplica hacia un grupo de niños donde el rumbo de la investigación es hacia la educación moral, componente importante para el desarrollo del ser humano, y que, se demostró que la sensibilidad moral mostró un incremento significativo, mas no el juicio moral. Y en otros estudios, \textcite{OJE2023100033} muestra la educaci{\'o}n asistida con RV, y con ello, que en un futuro se puedan desarrollar m{\'a}s contenidos hacia la educaci{\'o}n en ingenier{\'i}a en realidad virtual.

En tiempos actuales, se ha notado como la realidad virtual ha empezado a formar parte de muchas áreas del trabajo y de la vida, como lo pueden ser la medicina, entretenimiento, educación, etc. Numerosos estudios han utilizado la realidad virtual en la educación, y han desarrollado aplicaciones, por ejemplo, donde se apoya a los estudiantes a entender conceptos de la computación como lo puede ser un \textit{bubble sort}. \parencite{OYELERE2023100016} 

