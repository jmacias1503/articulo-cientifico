\section{Metodología}
Para este método se utilizo la cartografía conceptual, el cual “sirve para formar y evaluar conceptos esenciales de cada competencia en lo que respecta al saber conocer” \parencite[][p. 16]{tobon2012} 
\begin{table}[H]
   \caption{Tabla de artículos encontrados, revisados, y utilizados}
   \label{tab:articulos}
      \begin{tabular}{l|l l l}
         \hline
         ~ & Google académico & Elsevier & SciELO\\
         \hline
         Artículos encontrados & 258,000 & 1854 & 30\\
         \hline
         Artículos útiles & 7 & 13 & 2\\
         \hline
      \end{tabular}
\end{table}

Artículos de:

concepto metodología resultados innovación

\begin{table}[H]
   \caption{Artículos clasificados en su información}
   \label{tab:articulos-concepto-metod-res-inn}
      \begin{tabular}{|p{3.5cm}|p{3.5cm}|p{3.5cm}|p{3.5cm}|}
         \hline
         Concepto & Metodología & Resultados & innovación\\
         \hline
         \citetitle{rodriguez2021simulacion} & \citetitle{zhan2020augmented} & \citetitle{zamudio2021realidad} & \citetitle{barahona2019cospaces}\\
         \hline
         \citetitle{kaminska2019virtual} & \citetitle{palma2020realidad} & \citetitle{zheng1998virtual} &\\
         \hline
         \citetitle{elmqaddem2019augmented} & \citetitle{jang2021augmented} & \citetitle{chen2020effectiveness} &\\
         \hline
         \citetitle{ZAMMIT2023100035} & \citetitle{marin2022realidad} & \citetitle{OJE2023100033} &\\
         \hline
         \citetitle{YUDINTSEVA2023100018} & \citetitle{GUERRERO2022100002} & \citetitle{LOWELL2023100017} &\\
         \hline
         & \citetitle{RADU2023100011} & &\\
         \hline
         & \citetitle{SHIM2023100010} & &\\
         \hline
         & \citetitle{OYELERE2023100016} & &\\
         \hline
      \end{tabular}
\end{table}

\begin{table}[!h]
   \caption{Cartografía conceptual}
   \begin{tabular}{p{4cm}|p{4cm}|p{4cm}}
      Eje de análisis & Pregunta Central & Componentes\\
      \hline
      Noción & Como se puede dar cuenta &\\
      Categorización & &\\
      Caracterización & &\\
      Diferenciación & &\\
      Clasificación & En que rubros se puede clasificar las distintas aplicaciones de la RV&\\
      Vinculación & &\\
      Metodología & &\\
      Ejemplificación & Como se puede aplicar la RV en el ámbito de la educación? & Casos de aplicación de la RV en una muestra real\\
   \end{tabular}
\end{table}

\subsection{Cartografía Conceptual}

\subsubsection{Ejemplificación}

Un ejemplo aplicado de la RV en la educación son los estudios y experimentos hechos por \cite{SHIM2023100010}, donde se aplica hacia un grupo de niños donde el rumbo de la investigación es hacia la educación moral, componente importante para el desarrollo del ser humano, y que, se demostró que la sensibilidad moral mostró un incremento significativo, mas no el juicio moral.
