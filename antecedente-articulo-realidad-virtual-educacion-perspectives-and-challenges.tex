Este rápido avance en la realidad virtual, ha hecho que la forma de fabricar los dispositivos mejore y se adapte a las necesidades de las personas conforme pase el tiempo. En un principio se utilizaron tubos de rayos catódicos en paneles planos (LCD) y se han modificado a diodos orgánicos emisores de luz (OLED). Actualmente, estas tecnologías de visualización han dejado de limitarse a solamente paneles planos, y su enfoque es buscar revolucionar la forma en que se interactúa con el entorno. El objetivo principal en el desarrollo de estas tecnologías es poder lograr imágenes que se asemejen lo más posible a la realidad, sin llegar a causar incomodidad al usuario. Esto puede llegar a tener ciertas complicaciones en la parte tangible de la tecnología; como lo es el tamaño del dispositivo, el consumo de energía, la calidad de la imagen, el campo de visión, el brillo, entre otras características. Sin embargo, mientras se aborden estos desafíos, los avances en la tecnología óptica y experiencias de usuario pueden llegar a ser más envolventes y lleguen a ser utilizadas en diferentes campos laborales y educativos. \cite{zhan2020augmented}
