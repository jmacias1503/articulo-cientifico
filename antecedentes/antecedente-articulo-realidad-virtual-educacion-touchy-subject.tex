El futuro de la realidad virtual se centra en hardware, software, factores humanos y la entrega de experiencias. Por el lado del hardware, se necesitan dispositivos de retroalimentación háptica, pantallas de alta calidad y rastreadores 3D precisos. En el software, se requiere programación gráfica para la realidad virtual, solucionar problemas de interacción en 3D, trabajar en modelos de lenguaje como el reconocimiento de voz y diseñar sistemas operativos y de desarrollo adecuados. Los desafíos relacionados con factores humanos incluyen la efectiva utilización de entrada y salida en 3D, la incorporación de sonido y voz en las interfaces de usuario, y la investigación cognitiva para comprender la interacción humana en entornos virtuales. Además, la transmisión de realidad virtual a través de redes de alta velocidad implica la computación distribuida en tiempo real en procesadores paralelos a gran escala, lo que requiere estándares para conectar diferentes nodos de jugadores y definir contenido e interacción. \parencite{zheng1998virtual}
