Existen diferentes tipos de plataformas para utilizar la realidad virtual según los resultados y objetivos de aprendizaje que se desean. Estos objetivos pueden ser recordar y comprender, utilizar el conocimiento adquirido en situaciones típicas y/o utilizar el conocimiento adquirido en situaciones desafiantes. Estos objetivos pueden llegar a estar relacionados con el nivel de inmersión que se tenga. El primer tipo de plataforma de realidad virtual se utiliza para enseñar conocimientos teóricos en ciertos campos científicos. Estas herramientas no necesitan que exista una inmersión completa y pueden llegar a funcionar solamente con proyecciones en paredes, monitores o con gafas especiales. Estas plataformas se utilizan en visualización en 3D, aprendizaje en situaciones peligrosas o en viajes virtuales. El segundo tipo de plataforma se utiliza para enseñar habilidades prácticas basadas en conocimientos previos. Este tipo de aplicación necesita una inmersión más avanzada y un control más preciso. Para que esto suceda, son necesarios sensores externos. Para aumentar el realismo, se utiliza un casco de realidad virtual con seguimiento de movimiento. El tercer tipo de plataforma tiene como objetivo enseñar cómo aplicar conocimientos para solucionar problemas. Se utiliza principalmente en medicina e ingeniería. \cite{kaminska2019virtual}
