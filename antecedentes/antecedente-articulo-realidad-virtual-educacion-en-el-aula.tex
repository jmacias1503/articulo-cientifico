Esto nos lleva a la forma en que se puede incorporar la realidad virtual en los salones de clase, existen varios programas y aparatos que pueden servir como herramientas para mejorar la implementación de está tecnología. El profesor Carlos Barahona presenta una de estas herramientas llamada “CoSpaces Edu”. Este software da posibles soluciones a los principales problemas que pueden surgir a partir del uso de unas gafas de realidad virtual. Su programa está formado por un mundo abierto de coordenadas x, y, z; lo cual permite que exista un ambiente tridimensional además de poder generar objetos 3D modificables. Existe una versión gratuita de esta herramienta que permite hasta un acceso de 30 alumnos y es compatible con cualquier tipo de gafas de realidad virtual. El punto fuerte de sus sistema es la interacción con los objetos, lo cual logra usando ciertos lenguajes como Blocky, CoBlocks y Javascript. El objetivo de Carlos era poder responder si se podía utilizar VR en salones, con su herramienta ayuda a que los alumnos experimenten un entorno diferente al que están acostumbrados. Esto último puede llegar a ser útil en el proceso de aprendizaje debido a que no vuelve tan monótono la forma en que una persona obtiene conocimiento nuevo. \parencite{barahona2019cospaces}
