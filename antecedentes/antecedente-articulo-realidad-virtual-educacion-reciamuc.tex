Iniciando con el articulo de la “Simulación y realidad virtual aplicada a la educación” de la revista RECIAMUC; en donde nos presenta como la tecnología he estado avanzando en el ámbito de la educación tanto para alumnos como para docentes. Según este articulo, la realidad virtual esta dividida en los cuatro principales usos y aplicaciones de la realidad virtual; teniendo modulación, diseño, simulación y representación de la realidad. Estos principales usos tienen características en común que pueden ayudar a incentivar el proceso de aprendizaje en una persona, tales como la capacidad sintética, la interactividad, la interacción dinámica, el paseo virtual, la tridimensionalidad y los factores físicos. Aunque cada aspecto tiene todo un proceso por detrás, todos tiene el mismo fin, el cuál es el poder adaptarse a los diferentes estilos de aprendizaje. \parencite{rodriguez2021simulacion}
