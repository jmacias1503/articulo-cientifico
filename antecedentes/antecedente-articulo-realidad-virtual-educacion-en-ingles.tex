Las ventajas de la realidad virtual para la comunicación en ingles como segundo lenguaje ha probado que su inmersión, interacción, retroalimentación y creación han sede percibidas positivamente y son efectivas para lo que se afronta, la ansiedad, la motivación, confianza en si mismo, la conciencia cultural, la creatividad, y la voluntad de comunicarse. Aunque, los resultados en lo efectivo que puede ser el aprendizaje son aun inconclusos, aunque esto es probablemente causado por la gran carga cognitiva, problemas de equidad, experiencias no gratas, retos tecnológicos, y falta de actividades instrucciones en el ambiente de realidad virtual. \parencite{YUDINTSEVA2023100018}
