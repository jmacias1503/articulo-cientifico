Existe un cierto término conocido como e-learning, este se centra en los cambios que han ocurrido en los entornos de aprendizaje. Estos cambios muestran cómo es que el uso de plataformas o entornos de aprendizaje virtuales, están impulsando otras investigaciones relacionadas con medios y estrategias que apoyan la interacción entre participantes para el proceso de aprendizaje. Esto aborda aspectos como el diseño del plan de estudio, métodos de evaluación, tecnologías educativas, recursos de aprendizaje y promueve la colaboración interdisciplinaria en la investigación. Además, la gamificación, la realidad virtual, los sistemas digitales, la robótica, el Internet y simuladores, ayudan a hacer del entorno académico un lugar atractivo y motivador para los estudiantes. Es por esto mismo que el e-learning se enfoca en cómo las plataformas virtuales están transformando la forma en que aprendemos y cómo estas tecnologías pueden enriquecer la experiencia educativa de un estudiante. \cite{zamudio2021realidad}
