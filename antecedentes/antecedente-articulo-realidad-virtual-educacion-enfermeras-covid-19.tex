El cuerpo de enfermeros dedicados al aprendizaje continuo no solo fortalecen las bases de las organizaciones del cuidado de la salud, sino incrementan su experiencia hacia sus carreras profesionales también. Durante la crisis del coronavirus (COVID-19), estos jugaron un rol sumamente importante. Estudios previos han demostrado que las experiencias de simulación repetida mejoran las habilidades del pensamiento critico y técnico, y este ha sido utilizado a gran escala para mejorar la calidad del cuidado de la salud. Los escenarios utilizados en las sesiones de simulaciones de alta fidelidad eran sobre la clasificación de pacientes en la sospecha de haber contraído COVID-19, en donde, los participantes de estas sesiones tuvieron un pre-examen y un post-examen, antes y después de las sesiones respectivamente. Los hospitales deberían de invertir en establecer centros de simulación con Simulación de Alta Fidelidad y el adoptar la Simulación Virtual para el cuerpo de enfermeros y otros profesionistas del cuidado de la salud para su desarrollo de su entrenamiento. \parencite{GUERRERO2022100002}

\input{../tabla-guerrero2022-resultados-simulacion.tex}
