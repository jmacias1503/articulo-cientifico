\section{Introducción}

En este articulo se hablara de el uso de la realidad virtual, utilizado en el campo de la educación para mejorar esta, y salir de los esquemas del sistema tradicional.

A lo largo de las últimas décadas, el mundo ha sido testigo de una asombrosa evolución en el campo de las tecnologías, dando lugar a cambios revolucionarios en la forma en que vivimos, trabajamos y nos comunicamos. Desde los primeros días de la computación hasta la era actual de la inteligencia artificial y la conectividad global, las nuevas tecnologías han transformado radicalmente nuestra sociedad y han abierto las puertas a posibilidades inimaginables. Esta evolución ha sido impulsada por avances continuos en diversas áreas, como la informática, las comunicaciones, la electrónica, la educación, entre muchas otras. Para esta investigación se tratara sobre la educación y el impacto que puede tener una tecnología como la Realidad Virtual (RV) en este campo. La implementación de la RV en la educación no está exenta de desafíos. La inversión en hardware y software adecuados puede ser costosa, y la capacitación de educadores es esencial. El equilibrio entre la RV y la enseñanza tradicional también es crucial para un éxito sostenible.

Existen diversas herramientas que pueden ser utilizadas para mejorar la educación, y su implementación con la RV. Han habido diversas herramientas, e incluso empresas que han diseñado experiencias en RV con el propósito de determinar si son una opción sostenible y factible \parencite{SHIM2023100010}. Y esto no solo se limita a áreas especificas de la educación, también esto lo puede abarcar tanto la educación para infantes, medicina \parencite{GUERRERO2022100002}, e incluso el aprendizaje de lenguajes. \parencite{YUDINTSEVA2023100018, ZAMMIT2023100035}

%En algunas partes del mundo, todavía se cree que que los aparatos de realidad virtual y aumentada solo puede llegar al ámbito de los videojuegos, cuando todavía hay un enorme horizonte que se puede explorar en todos los ámbitos posibles, en esta ocasión, la educación desde todos los niveles, desde la básica hasta superior.

La investigación sobre este tema es debido a la tecnología que esta aumentando con creces, aparte que la realidad virtual es  una idea que se materializo apenas y es muy joven, e incluso con esto, ya se han mostrado beneficios utilizando en el ámbito de la educación, saliendo de los sistemas tradicionales de este.

Utilizar la realidad virtual como una herramienta de educación puede llegar a ser muy beneficioso, debido a las posibilidades que se puede llegar con esta, desde modelados 3D que se pueden visualizar con las perspectiva de quien la vea, hasta simulaciones con variables realistas en las cuales se pueden simular casos que llegan a suceder en la vida real, sin tener los riesgos de estar en una \parencite{chen2020effectiveness}.


%Una de los los motivos de investigación del tema ha sido la estructura del sistema educativo, en la cual, se llega a caer en muchas din\'amicas que son repetitivas, y, que si se exploraran de distintas maneras, se podría llegar a mejores resultados
En algunas partes del mundo, todavía se cree que que los aparatos de realidad virtual y aumentada solo puede llegar al ámbito de los videojuegos, cuando todavía hay un enorme horizonte que se puede explorar en todos los rubros posibles, en esta ocasión, la educación desde todos los niveles, desde la básica hasta superior.

El acercamiento a la realidad virtual ha sido muy lento, debido a supuestas complicaciones de salud que puede generar, al igual que los tabús que se tienen de esta. Adem\'as, en distintas profesiones en las que gran parte del aprendizaje que es empírico, en ciertos casos se aprende en situaciones de riesgo, un acercamiento desde la realidad virtual podría mantener es mismo aprendizaje, sin que hubiera peligro alguno.

